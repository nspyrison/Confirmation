% This is a LaTeX thesis template for Monash University.
% to be used with Rmarkdown
% This template was produced by Rob Hyndman
% Version: 6 September 2016

\documentclass{monashthesis}

%%%%%%%%%%%%%%%%%%%%%%%%%%%%%%%%%%%%%%%%%%%%%%%%%%%%%%%%%%%%%%%
% Add any LaTeX packages and other preamble here if required
%%%%%%%%%%%%%%%%%%%%%%%%%%%%%%%%%%%%%%%%%%%%%%%%%%%%%%%%%%%%%%%

\author{Nicholas S Spyrison}
\title{Dynamic visualization of high-dimensional functions via low-dimension
projections and sectioning across 2D and 3D display devices}
\degrees{B.Sc. Statistics, Iowa State University}
\def\degreetitle{Doctor of Philosophy}
% Add subject and keywords below
\hypersetup{
     %pdfsubject={The Subject},
     %pdfkeywords={Some Keywords},
     pdfauthor={Nicholas S Spyrison},
     pdftitle={Dynamic visualization of high-dimensional functions via low-dimension
projections and sectioning across 2D and 3D display devices},
     pdfproducer={Bookdown with LaTeX}
}


\bibliography{thesisrefs}

\begin{document}

\pagenumbering{roman}

\titlepage

{\setstretch{1.2}\sf\tighttoc\doublespacing}

\chapter*{Acknowledgements}\label{acknowledgements}
\addcontentsline{toc}{chapter}{Acknowledgements}

I would like to thank \dots

\chapter*{Declaration}\label{declaration}
\addcontentsline{toc}{chapter}{Declaration}

I hereby declare that this thesis contains no material which has been
accepted for the award of any other degree or diploma in any university
or equivalent institution, and that, to the best of my knowledge and
belief, this thesis contains no material previously published or written
by another person, except where due reference is made in the text of the
thesis.

\vspace*{2cm}\par\authorname

\chapter*{Preface}\label{preface}
\addcontentsline{toc}{chapter}{Preface}

The material in Chapter \ref{ch:intro} has been submitted to the journal
\emph{Journal of Impossible Results} for possible publication.

The contribution in Chapter \ref{ch:litreview} of this thesis was
presented in the International Symposium on Nonsense held in Dublin,
Ireland, in July 2015.

\chapter*{Abstract}\label{abstract}
\addcontentsline{toc}{chapter}{Abstract}

This thesis is about \ldots{}

\clearpage\pagenumbering{arabic}\setcounter{page}{0}

\chapter{Introduction}\label{ch:intro}

This is where you introduce the main ideas of your thesis, and an
overview of the context and background.

In a PhD, Chapter 2 would normally contain a literature review.
Typically, Chapters 3--5 would contain your own contributions. Think of
each of these as potential papers to be submitted to journals. Finally,
Chapter 6 provides some concluding remarks, discussion, ideas for future
research, and so on. Appendixes can contain additional material that
don't fit into any chapters, but that you want to put on record. For
example, additional tables, output, etc.

\chapter{Literature review}\label{ch:lit_review}

\section{Touring}\label{sec:tour}

\subsection{Overview}\label{overview}

In univariate datasets histograms, or smoothed density curves are
employed to visualize data. In bivariate data scatterplots and contour
plots (2-d density) can be employed. In three dimensions the two most
common techniques are: 2-d scatter plot with the 3rd variable as an
aesthetic (such as, color, size, height, \(etc.\)) or rendering the data
in a 3-d volume using some perceptive cues giving information describing
the seeming depth of the image
\footnote{Graphs of data depicting 3 dimension are typically printed on paper, or rendered on a 2-d monitor, they are intrinsically 2-d images. They are sometimes referred to as 2.5-d, or more frequently erroneously referred to as 3-d, more on this later.}.
When there are 4 variables: 3 variables as spatial-dimensions and a 4th
as aesthetic, or a scatterplot matrix consisting of 4 histograms, and 6
unique combinations of bivariate scatterplots.

Let \(p\) be the number of numeric variables; how do we visualize data
for even modest values of \(p\) (say 6 or 12)? It's far too common that
visualizing in data-space is dropped altogether in favor of modeling
parameter-space, model-space, or worse, long tables of statistics
without visuals\autocite{wickham_visualizing_2015}. Yet, we all know of
the risks and possible mis-leadingness of relying too heavily on
parameters alone\autocites{anscombe_graphs_1973}{matejka_same_2017}. So
why do we get aware from visualizing in data-space? Scalability, in a
word, we are not familiar with methods that allow us to concisely depict
and digest \(p \geq 5\) or so dimensions. This is where touring comes
in; using the wide range of touring techniques we are able to preserve
the visualization of data-space, and the intrinsic understanding of
structure and the data, beyond looking at statistic values alone.

Touring is a linear dimensonality reduction technique that orthagonally
projects \(p\)-space down to \(d\)-space. Many of these projections are
interpolated while varying the rotation of \(p\)-space and viewed in
order to the effect of watching an animation of the lower dimensional
embedding changing as \(p\)-space is manipulated. Shadow puppets offer a
useful analogy to aid in conceptualizing touring. Imagine a fixed light
source facing a wall. When a hand or puppet is introduced the
3-dimensional object projects a 2-dimensional shadow onto the wall. This
is a physical representation of a simple projection, that from \(p=3\)
down to \(d=2\). If the object rotates then the shadow correspondingly
changes. Observers watching only the shadow are functionally watching a
2-dimensional tour as the 3-dimensional object is manipulated.

\subsection{History}\label{history}

In 1974 Friedman and Tukey purposed Projection
Pursuit\autocite{friedman_projection_1974} (sometimes referred to as PP)
while working at Bell Labs. Projection Pursuit involves identifying
``interesting'' projection, remove a single component of the data, and
then iterate in this newly embedded subspace. Within each subspace the
projection seeks for a local extrema via gradient descent (historically
referred to as hill climbing algorithms), hence the nomenclature
pursuit.

Touring was first introduced by Asimov in 1985 with his purposed Grand
Tour\autocite{asimov_grand_1985} at Stanford University. In which,
Asimov suggested three types of Grand Tours: torus, at-random, and
random-walk. The specifics of which will be discussed below in the
Typology section.

\ldots{} Note that the the above methods have no imput from the user
aside from the starting basis. the remainder touring development has
largely been around user interaction and application.

Below is a non-exhaustive list of software implementing touring in some
degree, ordered by descending year:

\begin{itemize}
\tightlist
\item
  Spinifex\autocite{spinifex} -- 2018 for Linux, Unix, and Windows.
\item
  Tourr\autocite{wickham_tourr_2011} -- 2011 for Linux, Unix, and
  Windows. R package.
\item
  CyrstalVision\autocite{wegman_visual_2003} -- 2003 for Windows
\item
  GGobi\autocite{swayne_ggobi:_2003} -- 2003 for Linux and Windows
\item
  VRGobi\autocite{nelson_xgobi_1998} -- 1998 for use with the C2 in
  steroscopic 3d spaces
\item
  ExplorN\autocite{carr_explorn:_1996} -- 1996 for SGI Unix
\item
  XGobi\autocite{swayne_xgobi:_1991} -- 1991 for Linux, Unix, and
  Windows (via emulation)
\item
  XLispStat\autocite{tierney_lisp-stat:_1990} -- 1991 for Unix, and
  Windows
\item
  Prim-9\autocites{asimov_grand_1985}{fisherkeller_prim-9:_1974} -- 1985
  on internal os
\end{itemize}

additionally there have been a number of graphical user interfaces (gui)
that have been produced including{[}\textcite{huang_tourrgui:_2012};{]}

\subsection{Typology}\label{typology}

\subsubsection{Movement}\label{movement}

A fundamental aspect of touring is the path of rotation. Of which there
are four primary distinctions\autocite{buja_computational_2005}: random
choice, precomputed choice, data driven, and manual control.

\begin{itemize}
\tightlist
\item
  Random choice such as Asimov's grand tour\autocite{asimov_grand_1985}.

  \begin{itemize}
  \tightlist
  \item
    torus
  \item
    at-random
  \item
    random-walk
  \end{itemize}
\item
  Precomputed choice, \emph{e.g.} the little
  tour\autocite{mcdonald_interactive_1982}.

  \begin{itemize}
  \tightlist
  \item
    little tour
  \end{itemize}
\item
  Data driven - a guided tour performing (stochastic) gradient descent
  on some objective function\autocite{hurley_analyzing_1990}.

  \begin{itemize}
  \tightlist
  \item
    holes
  \item
    cmass
  \end{itemize}
\item
  Manual control, a constrained rotation on selected manipulation
  variable and magnitude\autocite{cook_manual_1997}.
\end{itemize}

\ldots{} * torus: where a \(p\)-dimensional torus, \(T^p\) is created
from a Cartesian product of \(p\) unit circles with
\(T^p \in \mathbb{R}^p\). Unfortunately uniformity of the parameters do
not correlate to uniform points on the surface of the torus. If step
distance between frames is fixed, disproportionate time is spent between
subspaces. If step distance is change to account for uniform points on
the torus then the continuity of the tour is lost.\\
* at-random: where each 2-frame is chosen at random without replacement.
This affords an assured uniform distribution of subspaces, but is far to
discontinuous for observation. It also leaves no parameters to control.
* random-walk: combines the continuity of the torus method and the
uniformity of the at-random method while leaving room for a control
parameter.

\subsubsection{Geoms}\label{geoms}

Scatterplots offer a simple, general case for viewing lower-dimension
embeddings of higher-dimensions. Such visualization offer \(p\)-dim down
to \(d\)-dim, typically two in the case of a standard monitor. Yet, no
intrinsic value stops touring being used in other graphics or geoms
(geometrics). For instance using parallel coordinate plots
(PCP)\autocite{ocagne_coordonnees_1885}, Andrews plots
\autocite{andrews_plots_1972}, Chernoff faces
\autocite{chernoff_use_1973}, anaglyphs \autocite{rollmann_zwei_1853},
star glyphs \autocite{siegel_surgical_1972}, scatterplot matrix
\autocite{chambers_graphical_1983}, and even pictures all offer
perfectly valid graphs in \(p\)-dimensions.

This works well when the number of dimensions being toured is small (in
the neighborhood of 5-10), yet the number of view, or 2-frames and we
can produce from \(p\)-space suffers from the so called blessing/curse
of dimensionality. In which the plethora of degrees of freedom either
offer many (non-unique) solutions to a problem or something that becomes
ever increasing unlikely, \(ie.\) a bivariate square needs .

\subsection{Linear vs non-linear dimensonality
reduction}\label{linear-vs-non-linear-dimensonality-reduction}

\section{Virtual reality}\label{virtual-reality}

\chapter{Tour methodology}\label{ch:tour}

\section{Spinnifex}\label{spinnifex}

\subsection{Tourr}\label{tourr}

\subsection{Plottly}\label{plottly}

\chapter{Display dimensionality}\label{ch:disp_dim}

\section{My work}\label{my-work}

\subsection{XGobbi vs the C2}\label{xgobbi-vs-the-c2}

\chapter{Human-computer interaction of 3d
projections}\label{ch:hci_3dproj}

\section{Tour in 3D}\label{tour-in-3d}

\subsection{ImAxes / IATK}\label{imaxes-iatk}

\appendix

\chapter{Additional stuff}\label{additional-stuff}

You might put some computer output here, or maybe additional tables.

Note that line 5 must appear before your first appendix. But other
appendices can just start like any other chapter.

\printbibliography[heading=bibintoc]



\end{document}
